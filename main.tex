\documentclass[10pt]{amsart}
\usepackage{lmodern}

\makeatletter
\ifcase \@ptsize \relax% 10pt
  \newcommand{\miniscule}{\@setfontsize\miniscule{4}{5}}% \tiny: 5/6
\or% 11pt
  \newcommand{\miniscule}{\@setfontsize\miniscule{5}{6}}% \tiny: 6/7
\or% 12pt
  \newcommand{\miniscule}{\@setfontsize\miniscule{5}{6}}% \tiny: 6/7
\fi
\makeatother

\usepackage{changepage}
%     If your article includes graphics, uncomment this command.
\usepackage{colortbl}
%\usepackage{graphicx}
\usepackage{leftidx}
\usepackage{mathtools}
\usepackage{graphicx}

\usepackage{expl3}
\ExplSyntaxOn
\cs_new_eq:NN \fpeval \fp_eval:n
\cs_new_eq:NN \clistitem \clist_item:Nn
\cs_new_eq:NN \foreachint \int_step_inline:nnnn
\ExplSyntaxOff

\usepackage{pgf}
\usepackage{pgf-spectra}
%\usepackage{fp}
%\usetikzlibrary{fpu}
%\usetikzlibrary{fpu}

%\pgfkeys{/pgf/fpu=true}

\newcommand{\eq}{=}
\newcommand{\setvalue}[1]{\pgfkeys{/variables/#1}}
\newcommand{\getvalue}[1]{\pgfkeysvalueof{/variables/#1}}
\newcommand{\declare}[1]{%
 \pgfkeys{
  /variables/#1.is family,
  /variables/#1.unknown/.style = {\pgfkeyscurrentpath/\pgfkeyscurrentname/.initial = ##1}
 }%
}
\pgfkeys{/pgf/number format/.cd ,precision=12,sci generic={exponent={\times 10^{#1}}}}
%\pgfset{fpu=true}


\usepackage{gnuplottex}
\usepackage{siunitx}
%\usepackage{subcaption}
%\usepackage{subfigure}
\usepackage{float}
\newcommand{\sinc}{\operatorname{sinc}}
\newcommand{\rect}{\operatorname{rect}}
\newcommand{\wll}{\textcolor{white}{123}}
\newcommand{\tot}{\text{tot}}
\newcommand{\ptl}{\text{ptl}}
\newcommand{\ove}{\operatorname{vec}}
\newcommand{\apr}{\text{apparatus}}
\newcommand{\dtr}{\text{dtr}}
\newcommand{\initial}{\text{initial}}
\newcommand{\final}{\text{final}}
\newcommand{\BS}{\operatorname{BS}}
\newcommand{\path}{\text{path}}
\newcommand{\up}{\uparrow}
\newcommand{\down}{\downarrow}
%\usepackage[dvipsnames]{xcolor}

%\usepackage[x11names]{xcolor}
\newtagform{blue}{\color{blue}(}{)}
%\usepackage[dvipsnames]{xcolor}

%\usepackage{pgfplots} 
 %   \usetikzlibrary{intersections}
    % use this `compat' level or higher so that TikZ coordinates don't have to be prefixed
    % with `axis cs:'
  %  \pgfplotsset{width=15cm,compat=1.11}
\usepackage{amsthm}
\usepackage{amsmath,amssymb}
\usepackage{helvet}
%\usepackage[leqno]{amsmath}
%\usepackage{blindtext}
\usepackage{bbm}

\makeatletter
\newcommand{\newparallel}{\mathrel{\mathpalette\new@parallel\relax}}
\newcommand{\new@parallel}[2]{%
  \begingroup
  \sbox\z@{$#1T$}% get the height of an uppercase letter
  \resizebox{!}{\ht\z@}{\raisebox{\depth}{$\m@th#1/\mkern-5mu/$}}%
  \endgroup
}
\makeatother



\DeclareSymbolFont{extraup}{U}{zavm}{m}{n}
\DeclareMathSymbol{\varheart}{\mathalpha}{extraup}{86}
\DeclareMathSymbol{\vardiamond}{\mathalpha}{extraup}{87}
\newcommand{\redheart}{\textcolor{red}{$\varheart$}}
\newcommand{\heart}{\ensuremath\varheart}

%\definecolor{orange}{rgb}{1.0, 0.7, 0}
\definecolor{awesome}{rgb}{1.0, 0.13, 0.32}
\definecolor{}{rgb}{0.0, 0.0, 0.60}
\makeatletter\newcommand{\leqnomode}{\tagsleft@true\let\veqno\@@leqno}
\newcommand{\reqnomode}{\tagsleft@false\let\veqno\@@eqno}\makeatother

\makeatletter
\newcommand{\pushright}[1]{\ifmeasuring@#1\else\omit\hfill$\displaystyle#1$\fi\ignorespaces}
\newcommand{\pushleft}[1]{\ifmeasuring@#1\else\omit$\displaystyle#1$\hfill\fi\ignorespaces}
\makeatother

\makeatletter
\newcommand{\specialcell}[1]{\ifmeasuring@#1\else\omit$\displaystyle#1$\ignorespaces\fi}
\makeatother

%\newcommand{\subsec}[1]{\begin{adjustwidth}{-0.1in}{0in}\Large\textbf{\textcolor{darkblue}{#1}}\end{adjustwidth}\normalsize \vspace{1ex}}

\newcommand{\linebreakc}{\textcolor{white}{bl}\\}
\usepackage[english]{babel}
\usepackage[utf8]{inputenc}
% use KoTeX package for Korean 
\usepackage{kotex}
% for the fancy \koTeX logo
\usepackage{kotex-logo}


\newcommand{\expp}[1]{\exp\left(#1\right)}
\newcommand{\expb}[1]{\exp\left[#1\right]}
\newcommand{\sinb}[1]{\sin\left[#1\right]}
\newcommand{\cosb}[1]{\cos\left[#1\right]}
\newcommand{\sinp}[1]{\sin\left(#1\right)}
\newcommand{\cosp}[1]{\cos\left(#1\right)}

\newcommand{\logb}[1]{\log\left[#1\right]}
\newcommand{\lnb}[1]{\ln\left[#1\right]}
\newcommand{\logp}[1]{\log\left(#1\right)}
\newcommand{\lnp}[1]{\ln\left(#1\right)}


\usepackage{hyperref}
\usepackage[nameinlink,capitalize]{cleveref}
\hypersetup{
    colorlinks=true,
    linkcolor=blue,
    filecolor=magenta,      
    urlcolor=blue,
    pdftitle={Homework 4 : Lindblad equation },
    bookmarks=true,
    pdfpagemode=FullScreen,
    }
\let\oldref\ref
\renewcommand{\ref}[1]{(\oldref{#1})} 
\def\equationautorefname~#1\null{(#1)\null}
\newcommand{\tagg}[1]{\stepcounter{equation}\tag{\theequation}\label{#1} }
\urlstyle{same}
\newcommand{\taggg}[1]{\tag{#1}\label{#1} }
%\usepackage{subfig}


%\numberwithin{equation}{section}

%\numberwithin{example}{section}
%\usepackage{subfig}

%\numberwithin{subfigure}{figure}

%\captionsetup[subfigure]{subrefformat=simple,labelformat=simple,listofformat=subsimple}
%\renewcommand\thesubfigure{(\alph{subfigure})}


\newtheorem{theorem}{Theorem}
\newtheorem*{theorem*}{Theorem}
\numberwithin{theorem}{section}

\newtheorem{lemma}{Lemma}
\newtheorem{proposition}{Proposition}

\newtheorem{exercise}{Exercise}
\newtheorem*{exercise*}{Exercise}
\crefname{exercise}{Exercise}{exercises}

\newtheorem{example}{Example}
\newtheorem*{example*}{Example}
\numberwithin{example}{section}
%\newtheorem{lemma}[lemma]{Lemma}
%\newtheorem{proposition}[theorem]{Proposition}
\newtheorem{corollary}[theorem]{Corollary}

%\newtheorem{exercise}[theorem]{Exercise}





\newtheorem{remark}[theorem]{Remark}
\newtheorem*{remark*}{Remark}
\newtheorem{problem}{Problem}
\newtheorem*{problem*}{Problem}

\newtheorem{innercustompro}{Problem}
\newenvironment{prob}[1]
  {\renewcommand\theinnercustompro{#1}\innercustompro}
  {\endinnercustompro}




\newtheorem{innercustomlem}{Lemma}
\newenvironment{lem}[1]
  {\renewcommand\theinnercustomlem{#1}\innercustomlem}
  {\endinnercustomlem}
  
\newtheorem{innercustomthm}{Theorem}
\newenvironment{thm}[1]
  {\renewcommand\theinnercustomthm{#1}\innercustomthm}
  {\endinnercustomthm}


\newenvironment{prf}
  {\begin{proof}[\textbf{\textcolor{magenta}{Proof}}\\]}
  {\end{proof}}
\makeatletter
  \renewcommand\@upn{\textit}
\makeatother



\newenvironment{solution}
  {\begin{proof}[\textbf{\textcolor{blue}{Solution}}\\]}
  {\end{proof}}
\makeatletter
  \renewcommand\@upn{\textit}
\makeatother

\newcommand{\lf}{\left}
\newcommand{\rg}{\right}
\theoremstyle{definition}
\newtheorem{definition}{Definition}
\newtheorem*{definition*}{Definition}

\newtheorem{innercustomdefi}{Definition}
\newenvironment{defi}[1]
  {\renewcommand\theinnercustomdefi{#1}\innercustomdefi}
  {\endinnercustomdefi}

%\newtheorem{example}{Example}

%\newtheorem{xca}[theorem]{Exercise}

\newtheorem{innercustomex}{Example}
\newenvironment{ex}[1]
  {\renewcommand\theinnercustomex{#1}\innercustomex}
  {\endinnercustomex}

\theoremstyle{remark}


%\newenvironment{proposition}[2][Proposition]{\begin{trivlist}
%\item[\hskip \labelsep {\bfseries #1}\hskip \labelsep {\bfseries #2.}]}{\end{trivlist}}


\newcommand{\abs}[1]{\left\lvert#1\right\rvert}

\definecolor{hotpink}{rgb}{1.0, 0, 0.6}
\definecolor{inter}{rgb}{0.4, 0.4, 1}
\definecolor{awesome}{rgb}{1.0, 0.13, 0.32}
\definecolor{darkblue}{rgb}{0.0, 0.0, 0.60}
\definecolor{darkgreen}{rgb}{0.0, 0.50, 0.0}
\definecolor{darkyellow}{rgb}{0.80, 0.80, 0.0}
\newcommand{\red}[1]{\textcolor{red}{#1}}
\newcommand{\rred}[1]{\textcolor{RubineRed}{#1}}
\newcommand{\dblue}[1]{\textcolor{darkblue}{#1}}
\newcommand{\dgreen}[1]{\textcolor{darkgreen}{#1}}
\newcommand{\blue}[1]{\textcolor{blue}{#1}}
\newcommand{\cyan}[1]{\textcolor{cyan}{#1}}
\newcommand{\magenta}[1]{\textcolor{magenta}{#1}}
\newcommand{\brown}[1]{\textcolor{brown}{#1}}
\newcommand{\dyellow}[1]{\textcolor{darkyellow}{#1}}
\newcommand{\am}[1]{\textcolor{Aquamarine}{#1}}

\newcommand{\purple}[1]{\textcolor{purple}{#1}}

\newcommand{\green}[1]{\textcolor{green}{#1}}
\newcommand{\orange}[1]{\textcolor{orange}{#1}}

\newcommand{\ot}{\otimes}
\newcommand{\rh}{\rho}
\newcommand{\tht}{\theta}
\usepackage{answers}
\usepackage{setspace}
\usepackage{graphicx}
\usepackage{enumerate}
\usepackage{multicol}
\usepackage{amssymb}
\usepackage{mathrsfs}
\usepackage[margin=1in]{geometry} 

\usepackage{braket}
\usepackage{CJKutf8}
\newcommand{\ob}[1]{\mkern 1.5mu\overline{\mkern-1.5mu#1\mkern-1.5mu}\mkern 1.5mu}
 
\newcommand{\N}{\mathbb{N}}
\newcommand{\iZ}{\mathbb{Z}}
\newcommand{\C}{\mathbb{C}}
\newcommand{\R}{\mathbb{R}}
\newcommand{\Q}{\mathbb{Q}}
\newcommand{\X}{\mathcal{X}}
\newcommand{\Y}{\mathcal{Y}}
\newcommand{\Z}{\mathbb{Z}}
\newcommand{\W}{\mathcal{W}}

\newcommand{\norm}[1]{\left\lVert#1\right\rVert}
\newcommand{\xra}[1]{\xrightarrow{#1}}
\newcommand{\E}{\mathbb{E}}
\newcommand{\V}{\operatorname{Var}} 
\newcommand {\tr} {\operatorname{Tr}}
\newcommand{\trp}[1]{\tr\left(#1\right)}
\newcommand{\trpp}[2]{\tr_{#1}\left(#2\right)}
\newcommand{\trb}[1]{\tr\left[#1\right]}

\newcommand*{\myfont}{\fontfamily{lmss}\selectfont}
\DeclareTextFontCommand{\fo}{\myfont}

%\usepackage{titlesec}

\makeatletter
%default definition of article.cls
%using \renewcommand instead of \newcommand
\renewcommand\part{%
   \if@noskipsec \leavevmode \fi
   \par
   \addvspace{4ex}%
   \@afterindentfalse
   \secdef\@part\@spart}

\def\@part[#1]#2{%
    \ifnum \c@secnumdepth >\m@ne
      \refstepcounter{part}%
      \addcontentsline{toc}{part}{\thepart\hspace{1em}#1}%
    \else
      \addcontentsline{toc}{part}{#1}%
    \fi
    {\parindent \z@ \raggedright
     \interlinepenalty \@M
     \normalfont
     \ifnum \c@secnumdepth >\m@ne
       \Large\bfseries \partname\nobreakspace\thepart
       \par\nobreak
     \fi
     \huge \bfseries #2%
     %%%\markboth{}{}\par}% removing redefinition of headings
     \par}%
    \nobreak
    \vskip 3ex
    \@afterheading}
\def\@spart#1{%
    {\parindent \z@ \raggedright
     \interlinepenalty \@M
     \normalfont
     \huge \bfseries #1\par}%
     \nobreak
     \vskip 3ex
     \@afterheading}
\makeatother

%\renewcommand{\partname}{Chapter}
%\newcommand{\part}{\part}
%\begin{comment}
%\titleformat{\part}
 % {\normalfont\myfont\huge\bfseries}
  %{\parttitlename\ \thepart}{20pt}{\huge}
%\end{comment}
  
  
\usepackage{etoolbox}
\patchcmd{\section}{\scshape}{\bfseries}{}{}
\makeatletter
\renewcommand{\@secnumfont}{\bfseries}
\makeatother  
  
%\renewcommand{\thesection}{\arabic{section}}  



\makeatletter
\def\@seccntformat#1{%
  \expandafter\ifx\csname c@#1\endcsname\c@section
  \thesection\hspace{2ex}
  \else
  \csname the#1\endcsname\quad
  \fi}
\makeatother

  
\makeatletter  
\renewcommand\partname{Chapter}
\renewcommand\part{\@startsection{part}{0}%
\z@{\linespacing\@plus\linespacing}{1\linespacing}%
{\myfont\bfseries\huge\raggedright}}  
\makeatother  
  
  

% http://joshua.smcvt.edu/latex2e/bs-at-startsection.html  
  
\makeatletter
\renewcommand\section{\@startsection{section}{1}%
{0pt}{.8\linespacing\@plus\linespacing}{.6\linespacing}%
{\LARGE\bfseries\color{black}}}
\makeatother

\makeatletter
\renewcommand\specialsection{\@startsection{specialsection}{1}%
{0pt}{.8\linespacing\@plus\linespacing}{.6\linespacing}%
{\LARGE\bfseries\color{hotpink}}}
\makeatother

\makeatletter
\renewcommand\subsection{\@startsection{subsection}{2}%
{0pt}{.8\linespacing\@plus.9\linespacing}{.7\linespacing}%
{\Large\bfseries\color{black}}}
\makeatother

\makeatletter
\renewcommand\subsubsection{\@startsection{subsubsection}{3}%
\z@{.6\linespacing\@plus.7\linespacing}{.4\linespacing}%
{\large\bfseries\color{black}}}
\makeatother

%https://tex.stackexchange.com/questions/268706/amsart-change-subsection-headings-from-boldface-to-smallcaps
%https://tex.stackexchange.com/questions/60437/newlines-after-section-headings-in-amsart
%http://ftp.ktug.org/tex-archive/macros/latex/required/amscls/doc/amsclass.pdf 의 Line 1175

%Since amsart.cls has
%\def\subsection{\@startsection{subsection}{2}%
%  \z@{.5\linespacing\@plus.7\linespacing}{-.5em}%
%  {\normalfont\bfseries}}
%you just add, in your document preamble (that is, before \begin{document}),
%\makeatletter
%\renewcommand\subsection{\@startsection{subsection}{2}%
%  \z@{.5\linespacing\@plus.7\linespacing}{-.5em}%
%  {\normalfont\scshape}}
%\makeatother

\newcommand{\subsec}[1]{\begin{adjustwidth}{-0.1in}{0in}\subsection{#1}\end{adjustwidth}}
\newcommand{\subsect}[1]{\hspace{-0.2in}\subsection{#1}}
\newcommand{\secc}[1]{\begin{adjustwidth}{-0.1in}{0in}\section{#1}\end{adjustwidth}}

\usepackage{array}
\preto\tabular{\setcounter{magicrownumbers}{0}}
\newcounter{magicrownumbers}
\newcommand\rownumber{\small\stepcounter{magicrownumbers}\arabic{magicrownumbers}}


\newcommand{\vb}{\vec{B}}
\newcommand{\vj}{\vec{j}}
\newcommand{\vn}{\vec{\nabla}}
\newcommand{\vr}{\mathbf{r}}
\newcommand{\vk}{\mathbf{k}}
\newcommand{\va}{\vec{A}}
\newcommand{\vl}{\vect{l}}
\newcommand{\vp}{\varphi}\newcommand{\hvp}{\hat{\varphi}}
\setcounter{section}{+0}

\newcommand{\hq}{\hat{Q}}
\newcommand{\hp}{\hat{\Phi}}

\newcommand{\ha}{\hat{a}}
\newcommand{\hN}{\hat{N}}
\newcommand{\ld}{\lambda}
\newcommand{\htp}{\hat{p}}

\newcommand{\om}{\omega}
\newcommand{\var}{\operatorname{Var}}
\newcommand{\vect}{\mathbf}
\newcommand{\nul}{\operatorname{Nul}}
\newcommand{\col}{\operatorname{Col}}
\newcommand{\row}{\operatorname{Row}}
\newcommand{\dg}{\dagger}
%    Blank box placeholder for figures (to avoid requiring any
%    particular graphics capabilities for printing this document).
\newcommand{\blankbox}[2]{%
  \parbox{\columnwidth}{\centering
%    Set fboxsep to 0 so that the actual size of the box will match the
%    given measurements more closely.
    \setlength{\fboxsep}{0pt}%
    \fbox{\raisebox{0pt}[#2]{\hspace{#1}}}%
  }%
}



\usepackage{bbm}
%     If your article inludes graphics, uncomment this command
\usepackage{svg}
\usepackage[super, square]{natbib}
%\usepackage[square]{natbib}

%\usepackage{biblatex}
%\bibliography{name.bib}
\svgsetup{inkscapelatex=false}
\usepackage{epstopdf}
\usepackage{url}
%\usepackage{circuitikz}

\usepackage{pgf}
\usepackage{graphicx}
\usepackage{pgfplots}

\usepackage{gnuplottex}
\usepackage{pgfplotstable}
\usepgfplotslibrary{external} 
\usetikzlibrary{external}
%\tikzexternalize[prefix=tikz/]


%\pgfplotsset{every axis/.append style={
                    %axis x line=middle,    % put the x axis in the middle
                    %axis y line=middle,    % put the y axis in the middle
                    %axis line style={<->,color=blue}, % arrows on the axis
                    %xlabel={$x$},          % default put x on x-axis
                    %ylabel={$y$},          % default put y on y-axis
            %}}
\pgfplotsset{every tick label/.append style={font=\tiny}}
\pgfplotsset{every x tick label/.append style={font=\tiny, yshift=0.5ex}}
\pgfplotsset{every y tick label/.append style={font=\tiny, xshift=0.5ex}}
\pgfplotsset{ /pgf/number format/precision=10}

\begin{comment}

\pgfplotsset{
    node near coord/.style={ % Style for activating the label for a single coordinate
        nodes near coords*={
            \ifnum\coordindex=#1\pgfmathprintnumber{\pgfplotspointmeta}\fi
        }
    },
    nodes near some coords/.style={ % Style for activating the label for a list of coordinates
        scatter/@pre marker code/.code={},% Reset the default scatter style, so we don't get coloured markers
        scatter/@post marker code/.code={},% 
        node near coord/.list={#1} % Run "node near coord" once for every element in the list
    }
}

\pgfplotsset{
    node near coord/.style={ % Style for activating the label for a single coordinate
        nodes near coords*={
            \ifnum\coordindex=#1\pgfmathprintnumber{\pgfplotspointmeta}\fi
        }
    },
    nodes near some coords/.style={ % Style for activating the label for a list of coordinates
        scatter/@pre marker code/.code={},% Reset the default scatter style, so we don't get coloured markers
        scatter/@post marker code/.code={},% 
        node near coord/.list={#1} % Run "node near coord" once for every element in the list
    }
}

\pgfplotsset{
    node near coord/.style args={#1/#2/#3}{% Style for activating the label for a single coordinate
        nodes near coords*={
            \ifnum\coordindex=#1 #2\fi
        },
        scatter/@pre marker code/.append code={
            \ifnum\coordindex=#1 \pgfplotsset{every node near coord/.append style=#3}\fi
        }
    },
    nodes near some coords/.style={ % Style for activating the label for a list of coordinates
        scatter/@pre marker code/.code={},% Reset the default scatter style, so we don't get coloured markers
        scatter/@post marker code/.code={},% 
        node near coord/.list={#1} % Run "node near coord" once for every element in the list
    }

}

\pgfplotsset{
    node near coord/.style args={#1/#2/#3}{% Style for activating the label for a single coordinate
        nodes near coords*={
            \ifnum\coordindex=#1  #2\fi
        },
        scatter/@pre marker code/.append code={
            \ifnum\coordindex=#1 \pgfplotsset{every node near coord/.append style=#3}\fi
        }
    },
    nodes near some coords/.style={ % Style for activating the label for a list of coordinates
        scatter/@pre marker code/.code={},% Reset the default scatter style, so we don't get coloured markers
        scatter/@post marker code/.code={},% 
        node near coord/.list={#1} % Run "node near coord" once for every element in the list
    }

}

\end{comment}





\pgfplotsset{
    node near coord/.style args={#1/#2/#3}{% Style for activating the label for a single coordinate
        nodes near coords*={
            \ifnum\coordindex=#1\pgfmathprintnumber{\pgfplotspointmeta}#2\fi
        },
        scatter/@pre marker code/.append code={
            \ifnum\coordindex=#1 \pgfplotsset{every node near coord/.append style=#3}\fi
        }
    },
    nodes near some coords/.style={ % Style for activating the label for a list of coordinates
        scatter/@pre marker code/.code={},% Reset the default scatter style, so we don't get coloured markers
        scatter/@post marker code/.code={},% 
        node near coord/.list={#1} % Run "node near coord" once for every element in the list
    }
}


\pgfplotsset{
    pin near coord/.style args={#1/#2/#3}{% Style for activating the label for a single coordinate
        scatter/@pre marker code/.append code={
           \ifnum 1=#3
            \ifnum\coordindex=#1  point meta=x, \node[pin={#2:\small\color{black} \SI{\small\pgfmathprintnumber{\pgfplotspointmeta}}{\volt}  }]{};\fi 
            \fi 
            \ifnum 2=#3
            \ifnum\coordindex=#1 point meta=x, \node[pin={[align=center]75:#2}]{};\fi
            \fi
        }
    },
    pins near some coords/.style={ % Style for activating the label for a list of coordinates
        scatter,
        scatter/@pre marker code/.code={},% Reset the default scatter style, so we don't get coloured markers
        scatter/@post marker code/.code={},% 
        pin near coord/.list={#1} % Run "pin near coord" once for every element in the list
    }
}
\pgfplotsset{select coords between index/.style 2 args={
    x filter/.code={
        \ifnum\coordindex<#1\def\pgfmathresult{}\fi
        \ifnum\coordindex>#2\def\pgfmathresult{}\fi
    }
}}
%\usetikzlibrary {math, fpu}
%\pgfkeys{/pgf/fpu = true}
\usepackage{csvsimple}
\usepackage{ifpdf}
\usepackage{sidecap}
\usepackage{float}
\usepackage[labelformat=simple]{subcaption}
\renewcommand{\thesubfigure}{(\scshape\Alph{subfigure})}
\captionsetup[subfigure]{position=top}
\begin{comment}
\usepackage[labelformat=simple]{subcaption}
\renewcommand\thesubfigure{(\alph{subfigure})}
(Note: Since parens is the default label format you will get double parentheses in sub-captions if you don't specify a different label format, e.g., simple.)
\end{comment}
%\renewcommand\thesubfigure{(\alph{subfigure})}
\newcommand{\squeezeup}{\vspace{-5mm}}
\newcommand{\squeezeupp}{\vspace{-8.5mm}}
\newcommand{\squeezeuppp}{\vspace{-10mm}}
%\usepackage{pgfplots}
   % \usetikzlibrary{intersections}
    % use this `compat' level or higher so that TikZ coordinates don't have to be prefixed
    % with `axis cs:'
   %\pgfplotsset{width=15cm,compat=1.11}

\usepackage{siunitx}\sisetup{parse-numbers=false}
\sisetup{parse-numbers=false}
\DeclareSIUnit\torr{torr}
\DeclareSIUnit\gauss{G}
\usepackage{amsmath}
\usepackage{amsthm,amssymb}

\makeatletter
\def\convertto#1#2{\strip@pt\dimexpr #2*65536/\number\dimexpr 1#1}
\makeatother

\usepackage{mhchem}%
\usepackage{chemformula}
\let\ce\ch
\usepackage{longtable}
\usepackage{multirow, makecell}
\usepackage{multicol}
\begin{comment}

\makeatletter
\renewenvironment{thebibliography}[1]
     {\begin{multicols}{2}[\section*{\refname}]%
      \@mkboth{\MakeUppercase\refname}{\MakeUppercase\refname}%
      \list{\@biblabel{\@arabic\c@enumiv}}%
           {\settowidth\labelwidth{\@biblabel{#1}}%
            \leftmargin\labelwidth
            \advance\leftmargin\labelsep
            \@openbib@code
            \usecounter{enumiv}%
            \let\p@enumiv\@empty
            \renewcommand\theenumiv{\@arabic\c@enumiv}}%
      \sloppy
      \clubpenalty4000
      \@clubpenalty \clubpenalty
      \widowpenalty4000%
      \sfcode`\.\@m}
     {\def\@noitemerr
       {\@latex@warning{Empty `thebibliography' environment}}%
      \endlist\end{multicols}}
\makeatother
%\url{tex.stackexchange.com/questions/20758/bibliography-in-two-columns-section-title-in-one/20761}
\end{comment}
%\renewcommand{\bibpreamble}{\begin{multicols}{2}}
%\renewcommand{\bibpostamble}{\end{multicols}}
\usepackage{helvet}

\usepackage{blindtext}
\usepackage{mathtools}
%\usepackage[x11names]{xcolor}
%\newtagform{blue}{\color{blue}(}{)}
%\usepackage[dvipsnames]{xcolor}

\definecolor{awesome}{rgb}{1.0, 0.13, 0.32}
\definecolor{darkblue}{rgb}{0.0, 0.0, 0.60} 


\newcommand{\re}{\operatorname{Re}}
%\newcommand{\im}{\operatorname{Im}}
\newcommand{\im}{\implies}

%\newcommand{\redheart}{\textcolor{red}{$\varheartsuit$}}
%\newcommand{\heart}{\ensuremath\varheartsuit}



\usepackage[english]{babel}
\usepackage[utf8]{inputenc}
% use KoTeX package for Korean 
\usepackage{kotex}






%\numberwithin{equation}{section}

\usepackage{answers}
\usepackage{setspace}

\usepackage{enumerate}
\usepackage{enumitem}
\usepackage{multicol}
\usepackage{mathrsfs}
\usepackage[margin=1in]{geometry} 
\usepackage{changepage}
\usepackage{braket}
\usepackage{CJKutf8}


\DeclareTextFontCommand{\fo}{\myfont}

\newcommand*{\helve}{\fontfamily{phv}\selectfont}
\DeclareTextFontCommand{\fohe}{\helve}

\makeatletter
\newcommand*{\rom}[1]{\expandafter\@slowromancap\romannumeral #1@}
\makeatother


\begin{comment}
\newcommand{\vb}{\vec{B}}
\newcommand{\vj}{\vec{j}}
\newcommand{\vn}{\vec{\nabla}}
\newcommand{\vr}{\mathbf{r}}
\newcommand{\vk}{\mathbf{k}}
\newcommand{\va}{\vec{A}}
\newcommand{\vl}{\vect{l}}
\newcommand{\vp}{\varphi}\newcommand{\hvp}{\hat{\varphi}}
\setcounter{section}{+11}

\newcommand{\hq}{\hat{Q}}
\newcommand{\hp}{\hat{\Phi}}

\newcommand{\ha}{\hat{a}}
\newcommand{\hN}{\hat{N}}
\newcommand{\ld}{\lambda}
\newcommand{\htp}{\hat{p}}
%\usepackage[T1]{fontenc}
%\usepackage{bold-extra}
\newcommand{\var}{\operatorname{Var}}
\newcommand{\vect}{\mathbf}
\newcommand{\nul}{\operatorname{Nul}}
\newcommand{\col}{\operatorname{Col}}
\newcommand{\row}{\operatorname{Row}}
\newcommand{\dg}{\dagger}
\end{comment}
\usepackage{fancyhdr}
\usepackage{regexpatch}

\begin{comment}
%%% let's patch the amsart macros
\makeatletter
\renewcommand{\sectionmark}[2]{%
  \ifnum#1<\@m
    \markboth{\thesection. #2}{\thesection. #2}%
  \else
    \markboth{#2}{#2}%
  \fi}
\xpatchcmd*{\@sect}
  {\@tocwrite}
  {\csname #1mark\endcsname{#2}{#7}\@tocwrite}
  {}{}
\makeatother
\end{comment}

\pagestyle{fancy}
\fancyhf{}
%\rhead{}
\lhead{ }
%\fancyhead[L]{\rightmark}
\fancyhead[R]{\textcolor{black}{\thepage}}

%\renewcommand{\subsectionmark}[1]{%
%  \markright{\MakeUppercase{\thesubsection.\ #1}}}%

%\renewcommand{\sectionmark}[1]{%
%\markboth{\thesection\quad #1}{}}
%\fancyhead{}
%\fancyfoot[L]{\leftmark}
%\fancyfoot{}
%\fancyfoot[CE,CO]{\leftmark}
\cfoot{\thepage}
\newlength\FHoffset
\setlength\FHoffset{0.12in}

\addtolength\headwidth{2\FHoffset}

\fancyheadoffset{\FHoffset}

\usepackage{layouts}



\setcounter{tocdepth}{3}% to get subsubsections in toc

\let\oldtocsection=\tocsection

\let\oldtocsubsection=\tocsubsection

\let\oldtocsubsubsection=\tocsubsubsection

\renewcommand{\tocsection}[2]{\hspace{0em}\oldtocsection{#1}{#2}}
\renewcommand{\tocsubsection}[2]{\hspace{1em}\oldtocsubsection{#1}{#2}}
\renewcommand{\tocsubsubsection}[2]{\hspace{2em}\oldtocsubsubsection{#1}{#2}}
\renewcommand{\contentsname}{Table of Contents}

\begin{document}
%\pgfset{fpu = true}
\usetagform{blue} \reqnomode
\newgeometry{left=0.9in, right=0.9in, top=1in, bottom=1in}
%\newgeometry{left=0.7in, right=0.7in, top=1in, bottom=1in}
\fancypagestyle{plain}{
  \fancyhf{}
  \lhead{ }
  \chead{}
  \rhead{\textcolor{black}{\thepage}}
}

\begin{center}
\linebreakc
\linebreakc
\linebreakc
\huge
%\textbf{\fo{ EXP 1.}}

\textbf{Pharmacology Summary}

\textcolor{white}{bl}\\

\Large

Gyunghee Han

\end{center}

\textcolor{white}{bl}\\

\normalsize

\renewcommand{\contentsname}{\normalfont \bfseries \Large Table of Contents}
\tableofcontents

\section{Pharmacokinetics}

\section{Absorption}
\subsection{Bioavailability \& Plasma Concentration}
\SI{100}{mg} 경구투여, bioavailability 10\%일때 plasma concentration?
\begin{align*}
    \frac{\SI{100}{mg} \times 0.1 }{\SI{2.5}{L}}
= \SI{4}{mg/L}    
\end{align*}
\subsection{흡수에 영향주는 요소 }

    \subsubsection{ 약물의 물리화학적 성질} 
    \subsubsection{생리적 성질 }
    \subsubsection{ 투여 경로 }
    \begin{enumerate}
        \item oral
        
        \item sublingual 
        장점: avoids 1st-pass metabolism, quick absorption, rapid onset
        
        단점 : only small dose, only lipid soluble
        
        \item rectal
        
        장점 : infants, vomiting, unconscious, 1st-pass avoided in 1/3 of dose
        
        단점 : erratic, unpredictable, inconvenient
        \item iv
        \item im
        \item sc
        
        drug deposited in subcutaenous fat tissue 
        
        ex) insulin, tuberculosis test, skin allergy test
        
        장점 : slower, more constant absorption \textit{\textbf{than IM }}, longer action (subcutatenous fat tissue is a reservoir of druc), self injectable
        
        단점 : \textit{most allergenic}
        \item inhalation
        
        장점 : quick absorption, rapid onset 
        
        단점 : direct delivery to heart 
        
        
        \item transdermal
        \item topical
    \end{enumerate}


\section{Distribution : \textit{reversible}, to extravascular}
Dispersion of drug throughout fluids and tissues of the 
body

the process of reversible transfer of a drug to and from the site of measurement (vascular compartment) and the peripheral tissues  (extra-vascular compartment)
\subsection{$V_d$ (Apparent) Volume of Distribution}
약물이 분포하는 가상의 용적

어떤 특정한 체액에서 농도가 관찰되었을 때 그만한 양을 담을 수 있는 시스템의
이론적인 부피 크기


\begin{align*}
\text{plasma 농도}
\ \ = \ \ \frac{\text{흡수량}}{  V_d} \ \ = \ \ \frac{\text{투여량 }\ \cdot \ \text{ bioavailability }}{ V_d }
\end{align*}
\SI{100}{mg} 경구투여, bioavailability 10\%, $\displaystyle V_d = \SI{10}{L}$일때 plasma 농도는?
\begin{align*}
\text{plasma 농도}
\ \ = \ \  
\frac{\SI{100}{mg}\ \cdot \ 0.1}{\SI{10}{L}}
\ \ = \ \ \SI{1}{mg/L}
\end{align*}
\begin{equation}
V_d \ \ = \ \ \frac{\text{흡수량 } A }{C_\text{plasma}}   \label{eq:AVdC} 
\end{equation}
\SI{10}{mg} IV투여, 혈중농도 측정시 \SI{50}{\micro\gram/L}였다.  $V_d$는? 
\begin{align*}
V_d \ \ = \ \ \frac{\SI{10}{mg} }{\SI{50}{\micro\gram/L}}     \ \ = \ \ \SI{200}{L}
\end{align*}
Plasma 내에 존재하는 약물의 양은? ($\displaystyle \neq $ 흡수량!!!) 

약물이 plasma내에만 존재하는 것이 아니므로!!!

\begin{align*}
\text{plasma내의 약물양 } A_\text{plasma}\ \ = \ \ C_\text{plasma} \ \cdot \ V_\text{plasma}    
\end{align*}
\begin{align*}
\SI{50}{\micro\gram/L} \ \cdot \ \SI{2.5}{L}    
\end{align*}

\section{Elimination $\displaystyle=$ Metabolism $+$ Excretion}
\subsection{Clearance 청소율}

\section{Metabolism : \textit{irreversible} }
\textit{\textbf{Irreversible}} transformation of parent compounds 
into daughter compounds

\subsection{Sites of Drug Metabolism}
Major Site : Liver

Secondary Sites : Intestine, Kidney (proximal), Lung
\subsection{hepatic clearance}

\begin{align*}
\text{hepatic clearance }
CL_H \ = \ Q_H \ \cdot  \ 
\frac{f_u\cdot CL_\text{int}}{Q_H \ + \ f_u\cdot CL_\text{int}}
\end{align*}
\subsection{Extraciton Ratio}
\begin{align*}
0 \ \leq\ \  
\text{Extraction Ratio}    \ = \ \frac{f_u\cdot CL_\text{int}}{Q_H \ + \ f_u\cdot CL_\text{int}} \ \ \leq  \ 1 
\end{align*}
\begin{align*}
\text{hepatic clearance} CL_H \ = \ Q_H \cdot \text{ Extraction Ratio }    
\end{align*}
\begin{enumerate}
    \item High 
\end{enumerate}
\section{Excretion : \textit{irreversible}}
\textbf{\textit{Irreversible transfer }} from the systemic circulation into 
\textbf{extracorporal} fluids (urine or bile)

Drugs that are excreted in the urine or bile are 
\textbf{\textit{hydrosoluble}}. 

• \textit{\textbf{Lipophilic}} drugs cannot be excreted before \textit{\textbf{metabolism into more polar} }
compounds


Sites of Excretion : Kidney, Liver
\subsection{Renal Excretion}
\subsubsection{Glomerular Filtration (사구체 여과)}
\subsubsection{Tubular Secrtion (근위 분비)}
 Predominant in the proximal tubule (근위 세뇨관)
 
  Mainly by transporters. E.g. P-glycoprotein, MRPs, OATs, OCTs, etc.
 
\subsection{Biliary Excretion}
By means of transport systems :


\section{Pharmacokinetic Parameters}
\subsection{Primary : F, $V_d$, CL}
\subsection{Secondary : $k_e$, $t_\frac{1}{2}$}

\section{Pharmacokinetic Model}
\subsection{Compartment Model}
\subsubsection{Central : blood, heart, lung, liver, kidney}
\subsubsection{Peripheral : fat tissue, muscle tissue }
\subsubsection{Special : CSF, CNS}

\subsection{$C_\text{plasma} - t$ curve}
\subsection{Elimination Kinetics}


\subsubsection{0th order elimination : 등속 }
 약물제거 속도(시간에 따른 약물양의 감
소정도)가 약물의 양(농도)과 무관하게
일정
\subsubsection{1st order elimination : $ \text{제거속도} \propto \text{약물농도}$ : 약물이 시간당 일정한 비율로 감소}

약물제거속도(시간에 따른 약물양의 감
소정도)가 일정하지 않고 약물의 양(농
도)에 비례


대부분의 약물이 elimination 이 1st
order elimination 

$t_\frac{1}{2}$이 일정

\subsubsection{Equations (1st order)}

\begin{align*}
C(t) \ = \ C(0) \cdot e^{-k\cdot t}
\end{align*}
\begin{align*}
\lnp{C(t)} \ 
= \ \lnp{C(0)} \ - \ k\cdot t 
\end{align*}
where $\displaystyle k = \text{elimination rate constant or fractional rate of elimination} $ [\si{h^{-1}}] 

Relationship between $k$ and $t_\frac{1}{2}$ : 
\begin{align*}
    k\cdot t \ = \ \lnp{C(0)} - \lnp{C(t)} = \lnp{\frac{C(0)}{C(t)} }
\end{align*}
Therfore, 
\begin{align*}
t_\frac{1}{2} \ = \ \frac{\ln2}{k}    
\end{align*}
\subsubsection{Rate of Elimination (1st order)}
\begin{align*}
\text{Rate of Elimination } (\si{mg/h})
\ \ = \ \ k (\si{h^{-1}}) \ \cdot \ A (\si{mg})
\end{align*}
where $A : \text{amount present in body}$ and $k : \text{elimination rate constant} $
\begin{align*}
\text{Rate of Elimination } (\si{mg/h})
\ \ = \ \ C (\si{mg/L}) \ \cdot \ CL (\si{L/h})    
\end{align*}
where $C : $ drug concentration and $CL : $ clearance 청소율 
\begin{align*}
\text{Rate of Elimination } (\si{mg/h})
\ \ &= \ \ Q (\si{L/h}) \ \cdot \ (C_\text{in} - C_\text{out}) (\si{mg/L}) \\
&= \ \ Q (\si{L/h}) \ \cdot \ C \ \cdot \ \underbrace{\frac{(C_\text{in} - C_\text{out})}{C}}_{E}  \\
&= \underbrace{Q \ \cdot \ E}_{CL} \ \cdot \ C \\
&= CL \ \cdot \ C 
\end{align*}
\subsubsection{$k$, $CL$, $V_d$ 관계 : $\displaystyle k = \frac{CL}{V_d}$ }
\begin{align*}
CL\cdot C = k \cdot A \\
\implies
k \ = \ \frac{CL\cdot C}{\blue{A}}     
= \frac{CL\cdot C}{\blue{V_d\cdot C} }
= \frac{CL}{V_d}
\end{align*}
in second equality we used the fact that $\displaystyle \text{흡수량 }A = V_d \cdot C_\text{plasma}$ by \cref{eq:AVdC} $\displaystyle V_d = \frac{\text{흡수량} A}{C_\text{plasma}} $
\subsubsection{AUC (Area Under the Curve) }
\begin{align*}
    AUC = \int_0^\infty dt\  C(t)
\end{align*}
\subsubsection{약물농도 증가속도  = infusion 속도 - 제거속도 }
약물농도증가속도 = infusion속도 - 제거속도 
= $R_0 - k\cdot A = R_0 - CL\cdot C$


Infusion 속도=제거속도인
시점에 약물농도
일정하게 유지됨


Plateau 도달시간
• Continuous infusion시 농도가 plateau에 도달하는 시간은 투여한
약물양이나 얼마나 자주 약물을 투여했는지 와는 무관하고 약
물의 반감기와 관련되어 있다.

\section{Multiple Dosage Regimen }
\subsection{Plasma Concentration $C_\text{plasma}(t)$ following multiple-dose regimen ( $C_{\max, N}$, $C_{\min,N}$) }

$\tau$의 시간간격으로 반복투여시 
\begin{align*}
C_{\max, N}
&= C_{\max, 1} \ \cdot \ 
\lf[1 + e^{-k\tau}+ e^{-2k\tau}+ e^{-3k\tau}+\cdots + + e^{-(N-1)k\tau} \rg]\\
&= \magenta{C_{\max, 1}} \ \cdot \
\frac{1-e^{-Nk\tau}}{1-e^{-k\tau}} \\
&=\magenta{\frac{Dose}{V_d}  } \ \cdot \
\frac{1-e^{-Nk\tau}}{1-e^{-k\tau}}
\end{align*}
(질문) magenta 왜 equal???
\begin{align*}
C_{\min, N} \ = \ C_{\max, N} \cdot e^{-k\tau}    
\end{align*}
\subsection{Steady State}

들어오는 속도 == 제거되는 속도 

rate in == rate out

농도가 일정하게 유지됨 

\subsubsection{항정상태에서 $C_{\max, ss} $ \& $C_{\min,ss}$ }

\begin{align*}
e^{-Nk\tau} = 0     
\end{align*}
(질문) 위 왜 성립??
\begin{align*}
C_{\max,N} \ =\  \frac{Dose}{V_d}\cdot\frac{1-e^{-Nk\tau}}{1-e^{-k\tau}}
\end{align*}
Steady state approximation : \blue{$e^{-Nk\tau}\simeq 0$ }
\begin{align*}
    C_{\max, SS} \ &=\   \frac{Dose}{V_d}\cdot
    \frac{ 1-  \underbrace{ \blue{e^{-Nk\tau}} }_{ \substack { \blue{\simeq 0}\\ \text{ by ss approx} }} } {1-e^{-k\tau}} \\
\     &\simeq \ \frac{Dose}{V_d}\frac{1}{1-e^{-k\tau}}
\end{align*}
\begin{align*}
    C_{\min, ss} = C_{\max, ss} \cdot e^{-k\tau}
\end{align*}
\subsubsection{항정상태의 Average Level}
\begin{align*}
C_{av, ss} \cdot \tau \ = \ AUC_\tau    \ = \ \frac{Dose}{CL}
\end{align*}
(질문) 왜 $Dose= CL\cdot AUC_\tau$ 성립??? 
\subsubsection{Accumulation Index}
최초투여대비 반복투여에 의해 항정상태에 이르렀을 때의 농도
(amount)의 증가분 
\begin{align*}
C_{\max, ss} = C_{\max,1}\ \cdot \ \underbrace{\frac{1}{1-e^{-k\tau}}}_{\text{Accumulation Index}}
\end{align*}
투여간격이 짧고 반감기 길수록 acculmulation 많이 됨

\section{Kinetics Following an ExtraVascular Dose} 
\section{Changes in Pharmacokinetic Parameters}
\subsection{Changes in $V_d$}
\subsection{Changes in $CL$}




\section{외워야할 식}
\begin{align*}
    k_e \ = \ \frac{CL}{V_d}
\end{align*}
\begin{align*}
    t_\frac{1}{2} \ = \ \frac{\ln2}{k_e}
\end{align*}
\begin{align*}
Dose \ = \ CL \cdot AUC    
\end{align*}
\begin{align*}
\frac{AR}{A}
\ = \ -\frac{1}{K_d}\cdot \lf( [R]_\text{tot}\rg) 
\end{align*}
Scatchard plot 으로 알 수 있는 것 : 친화도 (기울기), 총수용체개수(y절편), 수용체종류, 결합수  



염기성 약물은 염기성 환경에서 비이온화 형태라서 흡수 잘 된다. 

비이온화 형태 $\implies$ 흡수잘됨 

약알칼리 약물은 소장에서 비이온화도가 높으므로 소장에서 흡수가 잘된다. (jejunum pH7.5-8.0) 210510 2교시 약리학서론,약물수용체,효능 신찬영교수님 17pg (얌첵) 


CL의 단위 \si{mg/h}가 아니라 \si{L/h}  단위시간당 양이 아니라 \textit{단위시간당 \textbf{부피}}이다 !!!!


\section{0427 1}

\begin{equation}
    I \  = \ G_K \times (V_m - E_K)
\end{equation}
$G_K$ : 얼마나 많이 열려있는지 

activation curve : $G_K$ - membrane potential curve 

막전압-gated curve

activation될 확률 ($G_K$)가 전압이 depol되면 될수록 커진다. 

hyperpol쪽에서는 0,

depol쪽에서는 1이되는 

S자 모양의 함수이다. 


voltage-clamp 실험 : voltage에 따라 얼마나 activation 되는지 보는 실험 


2번은 관심있으면 보면됨 

3번은 잘 알아야함 


\end{document}
